\documentclass[12pt]{article}
\usepackage[utf8]{inputenc}
\usepackage[T1]{fontenc}
\usepackage[ngerman]{babel}
\usepackage{geometry}
\usepackage{enumitem}
\usepackage{amssymb}

\geometry{a4paper, margin=2.5cm}
\setlength{\parindent}{0pt}
\setlength{\parskip}{6pt}

\newlist{checkboxes}{itemize}{2}
\setlist[checkboxes]{label=$\square$}
\usepackage{pifont}
\newcommand{\cmark}{\ding{51}}%
\newcommand{\xmark}{\ding{55}}%
\newcommand{\done}{\rlap{$\square$}{\raisebox{2pt}{\large\hspace{1pt}\cmark}}%
\hspace{-2.5pt}}
\newcommand{\checked}{\rlap{$\square$}{\large\hspace{1pt}\xmark}}



\title{Forschungsseminar Digital Humanities}
\author{\copyright Vera Piontkowitz}
\date{Wintersemester 2025/2026}

% Um ein Element "anzukreuzen", folgendem Format folgen: \item[\checked] (also hinter \item noch Element [\checked] ergänzen

\begin{document}

\maketitle

\section*{1. Allgemeine Informationen} 
\subsection*{Titel des zu begutachtenden Beitrags:}
\subsection*{Autor:innen:}
\subsection*{Journal/Konferenz:}
\subsection*{Name u. Matrikelnummer der gutachtenden Person:}
\subsection*{Kurzzusammenfassung (30 - 50 Wörter):} 
\subsection*{Beurteilung der eigenen Expertise (30 - 50 Wörter):}


\section*{2. Themenpassung}

\begin{checkboxes}
    \item[\checked] Der Text ist exakt auf die Ausrichtung des Journals/der Konferenz abgestimmt.
    \item Der Text ist weitgehend im Einklang mit der Ausrichtung des Journals/der Konferenz.
    \item Der Text hat eine gewisse Relevanz für das Journal/die Konferenz, könnte jedoch spezifischer auf die Themen ausgerichtet sein.
    \item Der Text ist nur entfernt relevant für das Journal/die Konferenz und weicht deutlich von den Hauptthemen ab.
    \item Der Text steht in keinem Zusammenhang mit dem Fokus des Journals/der Konferenz.
\end{checkboxes}

\subsection*{Begründung (100 - 150 Wörter):}

\section*{3. Aufbau des Beitrags}

\begin{checkboxes}
    \item Der Beitrag hat eine nachvollziehbare Struktur.
    \item Einleitung, Hauptteil und Fazit sind klar erkennbar und funktional.
    \item Die Argumentationsführung ist schlüssig und folgt einem roten Faden. 
    \item Die einzelnen Abschnitte bauen logisch aufeinander auf. 
\end{checkboxes}

\subsection*{Begründung/Kommentar (100 - 150 Wörter):}

\section*{4. Inhaltliche Qualität}

\begin{checkboxes}
    \item Es werden klare Forschungsfragen/Arbeitshypothesen formuliert. 
    \item Es wird eine sinnvolle Eingrenzung der Thematik vorgenommen. 
    \item Die Forschungslage wird angemessen berücksichtigt und der Beitrag in den Fachdiskurs eingeordnet. 
    \item Die verwendeten Methoden werden angemessen vorgestellt. 
    \item Die Ergebnisse werden nachvollziehbar dargestellt. 
    \item Die Ergebnisse werden kritisch diskutiert und interpretiert.
    \item Die Schlussfolgerungen sind nachvollziehbar aus der Argumentation abgeleitet. 
    \item Das Fazit fasst die wesentlichen Erkenntnisse prägnant zusammen.
    \item Limitationen der eigenen Arbeit werden reflektiert.
    \item Die Argumentation wird durch geeignete Belege gestützt. 
\end{checkboxes}

\subsection*{Stärken der Arbeit (100 - 150 Wörter):}

\subsection*{Schwächen der Arbeit (100 - 150 Wörter):}

\section*{5. Bedeutung für Theorie und Praxis}

\begin{checkboxes}
    \item Herausragende Bedeutung
    \item Bedeutsam
    \item Nicht unbedeutend
    \item Eher schwache Bedeutsamkeit
    \item Von geringer Bedeutung
    \item Absolut keine Relevanz
\end{checkboxes}

\subsection*{Begründung (100 - 150 Wörter):}

\section*{6. Fragen an die Autor:innen}
\subsection*{Mind. 2 Fragen, 100 - 150 Wörter:}

\section*{7. Qualität der Darstellung (Aufbau, Sprache/Stil)}

\begin{checkboxes}
    \item Exzellent geschrieben
    \item Gut geschrieben
    \item Lesbar
    \item Sollte überarbeitet werden
    \item Noch erhebliche Arbeit erforderlich
    \item Inakzeptabel
\end{checkboxes}

\subsection*{Begründung (30 - 50 Wörter):}

\section*{8. Gesamturteil des eingereichten Beitrags}

\begin{checkboxes}
    \item Definitiv annehmen (sehr hohe Qualität) - grüne Ampel
    \item Annehmen (gute Qualität) - grüne Ampel
    \item Eher annehmen (noch akzeptable Qualität; geringfügige Überarbeitungen werden angeraten) - gelbe Ampel
    \item Eher ablehnen (geringe Qualität; massive Überarbeitungen sind angebracht) - gelbe Ampel
    \item Ablehnen (unbedeutende Arbeit) - graue Ampel
    \item Definitiv ablehnen (hat keinen Verdienst) - graue Ampel
\end{checkboxes}

\subsection*{Schlusskommentar (300 - 600 Wörter):}
\textit{(Bitte geben Sie hier eine Begründung, die es den Autor:innen erlaubt, Ihre Beurteilung nachzuvollziehen. Gehen Sie auf Stärken und Schwächen des Beitrages ein, eine Bezugnahme / Verweis auf obenstehende Kommentare zu Einzelaspekten ist möglich, der Kommentar sollte aber aus sich heraus verständlich sein.)}


\end{document}